\documentclass{article}
\usepackage{amsmath,amssymb,amsthm}
\usepackage{graphicx}
\usepackage{enumitem}
\newcommand{\KK}{\mathbb{K}}
\newcommand{\QQ}{\mathbb{Q}}

\begin{document}
    \section{Introduction}
    \begin{itemize}
        \item Thank you to the organizers for the opportunity to speak
        \item don't hesitate to interrupt to ask questions
        \item explain what Delta conjecture is and discuss some recent results obtained with Alessandro Iraci 
        \item this sentence contains three elements
    \end{itemize}

    \section{Symmetric functions}
    \begin{itemize}
        \item $e_n$ is the sum of all the products of $n$ distinct variables
        \item $e_2$ is an example of a symmetric function of homogeneous degree $2$, indeed all its monomials are of degree two
        \item it is easy to see that any symmetric function may be written as a finite sum of homogeneous symmetric functions
        \item in other words the space $\Lambda_{\KK}$ is graded by homogeneous degree
        \item the dimension of the space $\Lambda_\KK^{(n)}$ of symmetric functions of homogeneous degree $n$ is exactly the number of partitions of $n$
        \item a partition of $n$ is a decreasing vector of positive integers that sum to $n$
        \item the Ferrers diagram of a partition has as many boxes in the $i$-th row from the bottom as the $i$-th entry of the partition
        \item We discuss here some famous basis of $\Lambda_\KK^{(n)}$ that will be relevant to this talk.
        \item the monomial associated to a SSYT encodes its filling, that is the exponent of $x_i$ equals the number of $i$'s in the filling.
        \item If we take $\KK = \QQ(q,t)$ to be the smallest field containing to parameters $q$ and $t$ then the so-called \emph{Macdonald polynomials} are a remarkable basis of $\Lambda_\KK$. 
        \item Not only do they generalise other important families of symmetric functions (indeed for suitable choices of $q$ and $t$ we recover Schur functions, Hall-Littlewood polynomials and others), they seem to also have connections to Hilbert Schemes, Affine Hecke algebras and statistical physics.
    \end{itemize}

    \section{Interesting symmetric functions}
    \begin{itemize}
        \item why some symmetric functions are more interesting then others has to do with their connection to the representation theory of the symmetric group
        \item this connection is given by the Frobenius map which is an isomorphism between class functions of the $n$-th symmetric group and homogeneous symmetric functions of degree $n$
        \item it gives a correspondence between 
        \item irreducible representations (Specht modules) and schur functions
        \item Since, by Maschke's theorem, any finite dimensional representation is decomposable into irreducibles, this gives a correspondence between any representation of the symmetric group and positive integer combinations of schur functions. 
        \item If we now consider a bi-graded representation of the symmetric group and define a bi-graded version of the Frobenius characteristic map to keep track of the gradation as follows, \emph{then} bi-graded representations of the symmetric group correspond exactly to symmetric functions that can be written as a combination of schur functions with coefficients in the polynomial ring in the variables $q$ and $t$ and positive integer coefficients
        \item this is what we refer to as Schur positivity
        \item let us discuss some examples of interesting, that is schur positive symmetric functions
        \item since their introduction in the 80's, a version of Macdonald polynomials have been conjectured to be schur positive
        \item in the 90's Garsia and Haiman proved Macdonald positivity by showing that they are the Frobenius image of their Garsia-Haiman modules. They first reduced their proof to showing that the dimension of their modules equals $n!$, which Haiman proved in 2001. 
        \item In the process of proving Macdonald positivity, Garsia and Haiman introduced the space of diagonal coinvariants, a natural polynomial ring construction. 
        \item Its Frobenius characteristic is $\nabla e_n$, where $\nabla$ is an diagonal operator on Macdonald polynomials. (This was conjectured by Garsia Bergeron and proved by Haiman)
        \item Zabrocki introduced the space of \emph{super}diagonal coinvariants, and conjectured its Frobenius characteristic to be $\Delta'_{e_{n-k-1}}e_n$. 
        \item This is the symmetric function of the \emph{Delta conjecture}.
        \item $\Delta$ and $\Delta'$ are also diagonal operators on Macdonald polynomials that generalise $\nabla$ in a sense. 
    \end{itemize}

    \section{Combinatorics of lattice paths}
    \begin{itemize}
        \item a square path of size $n$ starts at $(0,0)$, ends at $(n,n)$, employs only unit north and east steps and ends with an east steps. Here we show a path of size $8$.
        \item A Dyck path is a square path that stays above the line $x=y$. 
        \item Since all Dyck paths are also square paths, I will give all the definitions on the larger set of square paths.
        \item the $0$'s are qualitatively different from the other labels.
        \item As for SSYT, the monomial of the path encodes the filling/labelling of it.
        \item in this definition, we set $x_0\mapsto 1$, so  it's as if the $0$'s are not present, hence \emph{partially labelled}.
        \item A \emph{contractible valley} is a vertical step preceded by another vertical step, which, if we were to delete it, would still yield a valid Dyck paths. In other words a contractible valley may not be preceded directly by a smaller label on the same diagonal.
    \end{itemize}

\section{Combinatorial formulas}

\subsection{Shuffle theorem}
    \begin{itemize}
        \item A combinatorial formula for $\nabla e_n$ in terms of labelled Dyck paths.
        \item Recall that $\nabla e_n$ is the Frobenius image of the diagonal coinvariants.
        \item This combinatorial formula was conjectured by Haglund et al. in 2005
        \item proved in 2018 by Carlson and Mellit
    \end{itemize}

\subsection{Delta conjecture}
\begin{itemize}
    \item The Delta conjecture was made by Haglund Remmel and Wilson in 2015
    \item it is an interpretation of $\Delta'_{e_{n-k-1}}e_n$ in terms of Dyck paths of size $n$ with $k$ decorations
    \item We present here the \emph{valley version} of the Delta conjecture, which is still an open problem. 
    \item The also exists a version where the decorations are on \emph{rises} instead of valleys. This version was proved very recently by D'Adderio and Mellit. 
    \item This proof made use of the novel Theta operators, introduced by Iraci, D'Adderio and myself. 
    \item We can rewrite the Delta conjecture symmetric function using Theta: we get $\Theta_k \nabla e_{n-k}$
    \item Comparing with the Shuffle theorem symmetric function, it seems that applying $\Theta_k$ has the effect of adding $k$ decorated steps to the combinatorics.
\end{itemize}

\subsection{Generalised Delta conjecture}
\begin{itemize}
    \item The generalised Delta conjecture was made by the same people in the same paper.
    \item It essentially asserts that applying $\Delta_{h_m}$ in the symmetric function side of the Delta conjecture, corresponds to adding $m$ steps labelled $0$ on the combinatorial side. 
\end{itemize}

\subsection{Square theorem}
\begin{itemize}
    \item Backing up in time, in 2007 Loehr and Warrington proposed a formula for $\nabla p_n$, up to a sign, in terms of labelled \emph{square} paths.
    \item It was proved by Sergel in 2018 to be a consequence of the Shuffle theorem.
\end{itemize}

\subsection{Generalised Delta Square theorem}
\begin{itemize}
    \item It was natural to look for a decorated, partially labelled equivalent of the generalised Delta conjecture on the square side. 
    \item The obvious interpretations did not work, but with the introduction of $\Theta$ and fiddling a bit with the combinatorial set, Iraci and I made the Generalised Delta square conjecture. 
\end{itemize}

\section{Two implications}
    \begin{itemize}
        \item These two implications put together makes the generalised Delta square conjecture conditional only on the Delta conjecture.
        \item For the remainder of the talk, I will talk a bit about the proof of these implications.
        \item I will mainly sketch the combinatorial ideas behind them and omit all the symmetric function manipulations.
    \end{itemize}

    \subsection{Delta $\Rightarrow$ generalised Delta}
        \begin{itemize}
            \item Start out with the set of paths for the Delta conjecture: i.e.\ labelled decorated Dyck paths. 
            \item We describe an algorithm to go from this set to the set of \emph{partially} labelled Dyck paths. In other words, we add $0$ labels.  
            \item The algorithm will allow us to keep track of the changes in statistics and labelling.
            \item Combined with a fancy identity reflecting the same behavior on the symmetric function side, this will yield a proof a the implication. 
        \end{itemize}

    \subsection{Delta $\Rightarrow$ Delta square}
        \begin{itemize}
            \item We follow the same general approach as Sergel used to prove that the square theorem follows from the shuffle theorem.
            \item build from scratch the set of square paths with a fixed set of labels in the diagonals
            \item during this process we keep track of the dinv (notice that for fixed labels in diagonals the area and monomial are constant)
            \item this allows us to get a factorization of the $q,t,x$ enumeration of the set, sometimes called a \emph{schedule formula}
            \item this formula, combined with some classical symmetric function identities, allows us to \emph{shift} all the labels one diagonal up
            \item thus we are able to go from square paths to Dyck paths.
            \item So the last thing I will show you all today is how to construct the set of paths with these labels in the diagonals.
            \item We start from the empty path and then add the biggest label, 3 into the diagonal containing the line $x=y$, which we will call the $0$-diagonal. 
            \item Then we add the smaller labels, two 2's into the same diagonal.
            \item there are three ways to do this, each creating a different amount of dinv
            \item  the dinv contributions will generally be counted by $q$-binomials, (for those who are familiar with them)
        \end{itemize}

\end{document}